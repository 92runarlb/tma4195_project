\documentclass[screen]{beamer}
\usepackage[T1]{fontenc}
\usepackage[utf8]{inputenc}
\usepackage{multimedia}

\usepackage{algorithm}
\usepackage{algorithmic}
\usepackage{bm}


% Bruk NTNU-temaet for beamer (her i bokmålvariant), alternativer er
% ntnunynorsk og ntnuenglish.
\usetheme{ntnuenglish}
 
% Angi tittelen, vi gir også en kortere variant som brukes nederst på
% hver slide:
\title[Eksempelforedrag]%
{COOL AND AWESOME TITLE}

% Denne kan du også bruke hvis det passer seg:
%\subtitle{Valgfri undertittel}

% Angir foredragsholder, også en (valgfri) kortversjon i
% hakeparanteser først som kommer nederst på hver slide:
\author[aut]{Aga, Kristian \\ Berge, Runar L. \\ Klemetsdal Øystein S.,\\
	Myrvoll Nilsen, Eirik \\ Selle, Maria.
}


% Institusjon. Bruk gjerne disse slik det passer best med det du vil
% ha.  Valgfri kortversjon her også
%\institute[NTNU]{Institutt for matematiske fag}

% Datoen blir også trykket på forsida. 
\date{November 23., 2015}
%\date{} % Bruk denne hvis du ikke vil ha noe dato på forsida.

% Fra her av begynner selve dokumentet
\begin{document}

% Siden NTNU-malen har en annen bakgrunn på forsida, må dette gjøres
% i en egen kommando, ikke på vanlig beamer-måte:
\ntnutitlepage

\section*{Modelling}
\begin{frame}
    \frametitle[jaddA]{Inundation}
    \begin{enumerate}[\label = $\circ$]
        \item    owje
        \item    Jee
        \item    Hm
    \end{enumerate}
    \begin{block}{This is a textbox with meaningful equations}
        \begin{align*}
            rhs = lhs.
        \end{align*}
    \end{block}
\end{frame}

\begin{frame}
    \frametitle{Inundation}
    \begin{itemize}
        \pause
        \item    h1
        \pause
        \item    h2    
    \end{itemize}
    \pause
    \begin{block}{This is another textbox with meaningful equations}
        \begin{align*}
            rhs = lhs.
        \end{align*}
    \end{block}
\end{frame}

\section*{Analysis}
\begin{frame}
    kk
\end{frame}

\section*{Numerical experiments}

\begin{frame}
    \frametitle{M(FD)$^2$S (Mimetic finite differences/forward difference) scheme}
    \pause
    
\end{frame}

\end{document}