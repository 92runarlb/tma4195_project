\documentclass[12pt]{article}
\usepackage{style}


\author{Aga, Kristian, Berge, Runar L., Klemetsdal Øystein,\\
	MyrvollNilsen, Eirik, Selle, Maria\\\\
	Industrial Mathematics\\
	NTNU \\\\
}
\title{Cool Title Goes Here \\     
	TMA4195}
\begin{document}
\maketitle
\clearpage

\section{Theory}
\subsection{Conservation of Mass}
Let $\rho(x,t)$ be the density of a fluid. We choose an arbitrary fixed
control volume
$\Omega$. Assuming no sources or sinks we get the conservation of mass on integral form
\begin{equation}
  \frac{\d }{\d t}\int\limits_{\Omega}\rho (\mathbf v, t) \d
  V + \int\limits_{\partial\Omega} \rho(\mathbf x,
  t)\ \mathbf{x\cdot n} \d S = 0.
\end{equation}
Using the divergence theorem we can transform the second integral into
a volume integral. Further we assume $\rho$ to be sufficiently smooth
so we can move the derivative inside the integral sign(Note that
$\Omega$ does not depend on time so $\frac{\d \rho(\mathbf x,t)}{\d
  t}= \frac{\partial \rho(\mathbf x, t)}{\partial t}$;
\begin{equation*}
  \int\limits_{\Omega}\left(\frac{\partial }{\partial t}\rho (\mathbf x, t) +
    \nabla\cdot(\rho(\mathbf x, t)\ \mathbf v) \right)\d V = 0.
\end{equation*}
Since the control volume we chose was arbitrary the integrand has to
be zero. This gives us the conservation of mass on differential form
\begin{equation}
  \frac{\partial }{\partial t}\rho (\mathbf x, t) +
    \nabla\cdot(\rho(\mathbf x, t)\ \mathbf v)  = 0.
\end{equation}
\subsection{Conservation of Momentum}
We now let the control volume $\Omega=\Omega(t)$ depend on time. 
By using Newton's second law of motion we can write the conservation
of momentum as
\begin{equation}
  \frac{\d }{d t}\int\limits_{\Omega(t)}\rho \mathbf v \d V =
  -\int\limits_{\partial \Omega}p\mathbf n \d S +
  \int\limits_{\Omega(t)}\rho\mathbf g\d V.
\end{equation}



\end{document}
