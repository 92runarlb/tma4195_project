\documentclass[12pt]{article}
\usepackage{style}


\author{Aga, Kristian, Berge, Runar L., Klemetsdal Øystein,\\
	MyrvollNilsen, Eirik, Selle, Maria\\\\
	Industrial Mathematics\\
	NTNU \\\\
}
\title{Cool Title Goes Here \\     
	TMA4195}
\begin{document}
\maketitle
\clearpage

% Chapter 1

\chapter{Introduction} % Main chapter title

\label{Chapter1} % For referencing the chapter elsewhere, use \ref{Chapter1} 

%----------------------------------------------------------------------------------------

% Define some commands to keep the formatting separated from the content 
\newcommand{\keyword}[1]{\textbf{#1}}
\newcommand{\tabhead}[1]{\textbf{#1}}
\newcommand{\code}[1]{\texttt{#1}}
\newcommand{\file}[1]{\texttt{\bfseries#1}}
\newcommand{\option}[1]{\texttt{\itshape#1}}

%----------------------------------------------------------------------------------------

In this modelling project, we will address three major phases in the development of a tsunami:
\begin{enumerate}[label = \emph{(\roman*)}]
    \item    The creation of a tsunami wave, when the kinetic energy of a falling rock is transferred to the
             water,
    \item    The propagation of the wave in the parts of the fjord where the bottom as approximately constant
             elevation,
    \item    The run-up of the wave when it reaches the end of the fjord and the height of the wave starts
             increasing at the same time that its speed decreases.
\end{enumerate}

In the case of a tsunami event close to a populated area, two questions are of particular interest:
\begin{enumerate}[label = \emph{(\roman*)}]
    \item    How long does it take for the wave to reach the populated area?
    \item    What is the height of the wave when it reaches the shore?
\end{enumerate}

Modeling tools in order to describe the problem, derive the modeling equa-
tions and finally simplify them by only retaining the processes that matter
for a given application (questions of type mod ).
(2) Analytical tools in order to solve simple equations and, in that way, gain
more insight into the governing equations (questions of type ana).
(3) Numerical tools in order to set up and implement numerical methods which
will enable you to treat more realistic cases (questions of type num).

\section{Theory}
\subsection{Conservation of Mass}
Let $\rho(x,t)$ be the density of a fluid. We choose an arbitrary fixed
control volume
$\Omega$. Assuming no sources or sinks we get the conservation of mass on integral form
\begin{equation}
  \frac{\d }{\d t}\int\limits_{\Omega}\rho (\mathbf v, t) \d
  V + \int\limits_{\partial\Omega} \rho(\mathbf x,
  t)\ \bm{x\cdot n} \d S = 0.
\end{equation}
Using the divergence theorem we can transform the second integral into
a volume integral. Further we assume $\rho$ to be sufficiently smooth
so we can move the derivative inside the integral sign(Note that
$\Omega$ does not depend on time so $\frac{\d \rho(\mathbf x,t)}{\d
  t}= \frac{\partial \rho(\mathbf x, t)}{\partial t}$;
\begin{equation*}
  \int\limits_{\Omega}\left(\frac{\partial }{\partial t}\rho (\mathbf x, t) +
    \nabla\cdot(\rho(\mathbf x, t)\ \mathbf v) \right)\d V = 0.
\end{equation*}
Since the control volume we chose was arbitrary the integrand has to
be zero. This gives us the conservation of mass on differential form
\begin{equation}
  \frac{\partial }{\partial t}\rho (\mathbf x, t) +
    \nabla\cdot(\rho(\mathbf x, t)\ \mathbf v)  = 0.
\end{equation}
\subsection{Conservation of Momentum}
We now let the control volume $\Omega=\Omega(t)$ depend on time. 
By using Newton's second law of motion we can write the conservation
of momentum as
\begin{equation}
  \frac{\d }{d t}\int\limits_{\Omega(t)}\rho \mathbf v \d V =
  -\int\limits_{\partial \Omega}p\mathbf n \d S +
  \int\limits_{\Omega(t)}\rho\mathbf g\d V.
\end{equation}

\section{Conservation of Mass}

We consider a fluid occupying a domain $\Omega \subset \mathbb{R}^3$. Let $\rho = \rho(\bm{x}, t)$, and
$\bm{v} = \bm{v}(\bm{x},t)$ denote the density and velocity of the fluid, respectively, at a position
$\bm{x} = (x,y,z) \in \Omega$ at a time $t > 0$. We look at a control volume $R$, with surface $\partial R$.
The rate of change of total mass in $R$ is then
\begin{align}
    \frac{d}{d t}\int_R \rho(\bm{x},t) \, d \bm{x}.
\end{align}
Consider now a small part of the surface $\partial R$, with area $ds$. Let $\bm{n}$ denote the outward normal
of $\partial R$. The mass flux through this surface element is then $-\rho \bm{v} \cdot \bm{n} \, ds$, so the
total mass flux through the boundary of $R$ is then
\begin{align}
    -\int_{\partial R}\rho \bm{v} \cdot \bm{n} \, ds.
\end{align}
Assuming that there are no sources or sinks within $R$, conservation of mass requires that
\begin{align}
    \frac{d}{d t}\int_R \rho(\bm{x},t) \, d \bm{x}
                            = -\int_{\partial R}\rho \bm{v} \cdot \bm{n} \, ds.
\end{align}
Using the divergence theorem, and assuming that $\rho$ is smooth, so that the derivative can be moved inside
the integral, we obtain
\begin{align}
    \int_R \big(\rho_t(\bm{x},t) + \nabla \cdot (\rho \bm{v}) \big) \, d \bm{x} = 0.
\end{align}
Now, this is true for any control volume $R \subset \Omega$. This gives that the integrand is zero. Indeed;
if we assume that the integrand is zero at some point in $\Omega$, continuity of $\rho$ and $\bm{v}$
gives that the integral would be positive, which is a contradiction. Hence, we have
\begin{align}
    \rho_t(\bm{x},t) + \nabla \cdot (\rho \bm{v}) = 0.
\end{align}

\section{Conservation of Momentum}

We have
\begin{align}
	\begin{aligned}
	    \frac{\d}{\d t} \int_{\Omega(t)} \rho \bm{v} \, \d V 
	        & = \int_{\Omega(t)} \big(\frac{\partial}{\partial t}(\rho \bm{v})
	                + \bm{v} \nabla \cdot (\rho \bm{v}) \big) \, \d V
	                - \int_{\partial \Omega} \bm{v} \times (\rho \bm{v}) \, \d S \\
	        & = \int_{\Omega(t)} \big(\frac{\partial}{\partial t}(\rho \bm{v})
	                + \bm{v} \nabla \cdot (\rho \bm{v}) \big) \, \d V                
	\end{aligned}
\end{align}

Div on each component
\begin{align}
    \int_{\partial \Omega} p \, \bm{n} \, d S = \int_{\Omega} \nabla p \, d V,
\end{align}

\begin{align}
    \int_{\Omega(t)} \bigg(\frac{\partial}{\partial t}(\rho \bm{v})
	                + \bm{v} \nabla \cdot (\rho \bm{v}) + \nabla p - \rho \bm{g} \bigg) \, \d V = 0
\end{align}

\begin{align}
    \lim_{\Omega(t) \rightarrow 0} \frac{1}{|\Omega(t)|}\int_{\Omega(t)} \bigg(\frac{\partial}{\partial t}(\rho \bm{v})
	                + \bm{v} \nabla \cdot (\rho \bm{v}) + \nabla p - \rho \bm{g} \bigg) \, \d V \\
	                = \frac{\partial}{\partial t}(\rho \bm{v})
	                + \bm{v} \nabla \cdot (\rho \bm{v}) + \nabla p - \rho \bm{g} = 0
\end{align}

Since $\phi$ is a scalar function, we have
\begin{align}
    \nabla \times (\nabla \phi) = \nabla \times (\phi_x, \phi_y, \phi_z) =
        \big(\phi_{zy} - \phi_{yz}, -(\phi_{zx} - \phi_{xz}), \phi_{xy} - \phi_{yx} \big).
\end{align}
From the properties of the mixed partial derivative, we have $\phi_{x_i x_j} = \phi_{x_j x_i}$. Hence,
\begin{align}
    \nabla \times (\nabla \phi) = 0 \quad \forall \text{ scalar functions } \phi.
\end{align}

\section{Boundary Conditions}

\begin{align}
	\begin{cases}
	    v\cdot \bm{n} = 0,    &  \quad  \forall (x,y) \in \mathbb{R}^2, z = -h(x,y) \\
	    \eta(x,y,0) = f(x,y), &  \quad  \forall (x,y) \in \mathbb{R}^2
	\end{cases}.
\end{align}


\end{document}
