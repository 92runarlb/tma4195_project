\documentclass[11pt]{article}
\usepackage{style}


\author{Aga, Kristian \\ Berge, Runar L. \\ Klemetsdal Øystein S.,\\
	Myrvoll Nilsen, Eirik \\ Selle, Maria\\\\
	Industrial Mathematics\\
	NTNU \\\\
}
\title{
TMA4195 \\ \vspace{5pt}
\hrule \vspace{10pt}
       {\Huge \textbf{Three Numerical Schemes for Simulating Waves}}
\vspace{10pt}\hrule\vspace{1cm}
}
\begin{document}
\maketitle
\vspace{1cm}
\hrule
\vspace{0.5cm}
\begin{abstract}
%	In this project we model a tsunami from a landslide at Åkerneset into Geirangerfjorden. Our main goal is to estimate the size of the tsunami
%	wave and the time it takes to reach the populated area Hellesylt. The development of the wave is studied in three phases; the creation of the wave,
%	the propagation and finally the run-up of the wave. \emph{Noen som kan skriver om de numeriske metodene.}
%	The main result of this project is that the wave uses about 3.5 minutes to reach Hellesylt and will be about 20 m high at this point. 
%	
	The tsunami from a landslide at Åkerneset into Geirangerfjorden has been studied using analytical tools, mathematical modeling and
	numerical analysis. The energy transferred from the landslide to the water was found to have lower and upper bounds of $8\times 10^{11}$ J and $1 \time 10^{14}$ J,
	respectively. The wave reached Hellesylt after 260 s with a height of 18 m, and Geiranger after 560 s with a height of 12 m. The run-up in Geiranger was found to
	be 60 m.
\end{abstract}
\thispagestyle{empty}
\clearpage
\tableofcontents
\thispagestyle{empty}
\clearpage
\setcounter{page}{1}
%
%    INTRODUCTION
%
\section{Introduction}
The unstable rock slide at Åkerneset, and the resulting tsunami wave, is an announced disaster, and a well-studied problem \citep{harbitz14}, \citep{lothe10}. In such an event, we are particularly interested in the time it
takes before the wave reaches the populated areas Hellesylt and Geiranger, and the height of the wave as this happens. In \cite{harbitz14}, the wave reaches Hellesylt after 270 s,
with a height of 40 m, while it reaches Geiranger after 580 s, with a height of 20 m. Moreover, the run-up is found to be between 35-85 m in Hellesylt, and 30 - 70 m in Geiranger.

In this report, we analyze the problem in three phases; the creation of the tsunami, the propagation of the wave and finally the run-up of the wave.
We will give estimates for this by using methods from mathematical modeling, analysis and numerical analysis.

\section{Governing Equations}

\subsection{Conservation of Mass}

We consider a fluid occupying a domain $\Omega \subset \mathbb{R}^3$. Let $\rho = \rho(\bm{x}, t)$, and
$\bm{v} = \bm{v}(\bm{x},t)$ denote the density and velocity of the fluid, respectively, at a position
$\bm{x} = (x,y,z) \in \Omega$ and time $t > 0$. We look at a control volume $\mathcal{C}$, with surface $\partial \mathcal{C}$.
The rate of change of total mass in $\mathcal{C}$ is then
\begin{align*}
    \frac{\d}{\d t}\int_\mathcal{C} \rho(\bm{x},t) \, \d V.
\end{align*}
Consider now a small part of the surface $\partial \mathcal{C}$, with area $\d S$. Let $\bm{n}$ denote the outward normal
of $\partial \mathcal{C}$. The mass flux through this surface element is then $-\rho \bm{v} \cdot \bm{n} \, \d S$, so the
total mass flux through the boundary of $\mathcal{C}$ is then
\begin{align*}
    -\int_{\partial \mathcal{C}}\rho \bm{v} \cdot \bm{n} \, \d S.
\end{align*}
Assuming that there are no sources or sinks within $\mathcal{C}$, conservation of mass requires that
\begin{align*}
    \frac{\d}{\d t}\int_\mathcal{C} \rho(\bm{x},t) \, \d V
                            = -\int_{\partial \mathcal{C}}\rho \bm{v} \cdot \bm{n} \, \d S.
\end{align*}

Using the divergence theorem we can transform the second integral into
a volume integral. Further we assume $\rho$ to be sufficiently smooth
so we can move the derivative inside the integral sign (Note that
$\mathcal{C}$ does not depend on time, so that $\frac{\d \rho(\bm x,t)}{\d
  t}= \rho_t(\bm x, t)$);
\begin{align*}
    \int_\mathcal{C} \big(\rho_t(\bm{x},t) + \nabla \cdot (\rho \bm{v}) \big) \, \d V = 0.
\end{align*}
Now, this is true for any control volume $\mathcal{C} \subset \Omega$. This gives that the integrand is zero. Indeed;
if we assume that the integrand is positive at some point in $\Omega$, continuity of $\rho$ and $\bm{v}$
gives that the integral would be positive, which is a contradiction. Hence, we have
\begin{align}
    \label{eq:massConservation}
    \rho_t(\bm{x},t) + \nabla \cdot (\rho \bm{v}) = 0.
\end{align}
%
%    Conservation of Momentum
%
\subsection{Conservation of Momentum}
Given a fluid volume $\Omega(t)$, containing the same water particles, we have that the exterior forces are
due to the pressure forces exerted by the particles outside $\Omega(t)$ and the gravitation. Hence, Newton's second
law yields
\begin{align*}
    \frac{\d }{\d t}\int_{\Omega(t)}\rho \bm{v} \, \d V = -\int_{\partial \Omega} p \bm{n} \, \d S + \int_{\Omega(t)} \rho \bm{g} \, \d V.
\end{align*}
Leibnitz rule for differentiation under the integral sign yields
\begin{align*}
	\begin{aligned}
	    \frac{\d}{\d t} \int_{\Omega(t)} \rho \bm{v} \, \d V 
	        & = \int_{\Omega(t)} \bigg(\frac{\partial}{\partial t}(\rho \bm{v})
	                + \bm{v} \nabla \cdot (\rho \bm{v}) \bigg) \, \d V
	                - \int_{\partial \Omega} \bm{v} \times (\rho \bm{v}) \, \d S 
	         = \int_{\Omega(t)} \bigg(\frac{\partial}{\partial t}(\rho \bm{v})
	                + \bm{v} \nabla \cdot (\rho \bm{v}) \bigg) \, \d V.           
	\end{aligned}
\end{align*}
Using the divergence theorem, for $p$, we obtain
\begin{align*}
    \int_{\Omega(t)} \bigg(\frac{\partial}{\partial t}(\rho \bm{v})
	                + \bm{v} \nabla \cdot (\rho \bm{v}) + \nabla p - \rho \bm{g} \bigg) \, \d V = 0.
\end{align*}
Moreover, assuming that $\rho$ is constant, we get
\begin{align*}
    \lim_{\Omega(t) \rightarrow 0} \frac{1}{|\Omega(t)|}\int_{\Omega(t)} \bigg(\frac{\partial}{\partial t}(\rho \bm{v})
	                + \bm{v} \nabla \cdot (\rho \bm{v}) + \nabla p - \rho \bm{g} \bigg) \, \d V 
	                = \frac{\partial}{\partial t}(\rho \bm{v})
	                + \bm{v} \nabla \cdot (\rho \bm{v}) + \nabla p - \rho \bm{g} = 0
\end{align*}
Hence, conservation of momentum yields
\begin{align}
    \label{eq:momentumConservation}
    \rho \bm{v}_t + \rho(\bm{v} \cdot \nabla) \bm{v} + \nabla p - \rho \bm{g} = 0.
\end{align}
%
%    Simplifying the equations
%
\subsection{Simplifying the equations}
In order to simplify the equations, we assume that the flow is irrotational, $\nabla \times \bm{v} = 0$. Since
our domain $\Omega$ is simply connected, Stokes theorem gives that
\begin{align*}
	\int_{\partial \mathcal{C}} \bm{v} \d s = \int_{\mathcal{C}} \nabla \times \bm{v} \, \d \bm{x},
\end{align*}
so the flow velocity is conservative. Thus the flow velocity is the function of the gradient
of some potential $\phi$.

We also have the vector calculus identity
\begin{align}
    \label{eq:vecCalc}
    \frac{1}{2}\nabla |\bm{w}|^2 = (\bm{w}\cdot \nabla) \bm{w} + \bm{w} \times (\nabla \times \bm{w}).
\end{align}

Inserting $\bm{v} = \nabla \phi$, into \eqref{eq:vecCalc}, we obtain $(\nabla \phi \cdot \nabla) \nabla \phi = \frac{1}{2}\nabla |\nabla \phi|^2.$ Further, we can write $\bm{g} = (0,0,g)$, where $g = |\bm{g}|$ is the gravitational acceleration, so that
$\bm{g} = \nabla (gz)$. Hence, from \eqref{eq:momentumConservation}, we have
\begin{align*}
    \nabla\bigg(\phi_t + \frac{1}{2}|\nabla \phi|^2 + \frac{p}{\rho} - g z \bigg) = 0
    \iff \phi_t + \frac{1}{2}|\nabla \phi|^2 + \frac{p}{\rho}  g z = C(t),
\end{align*}
where $C(t)$ is a function of $t$ alone. Assuming the water is undisturbed far away from the wave
%\begin{align*}
%    \begin{cases}
%	    \lim_{\bm{x} \rightarrow \pm(\infty,\infty, 0)}\bm{v} & = 0 \\
%	    \lim_{\bm{x} \rightarrow \pm(\infty,\infty, 0)}\eta   & = 0
%	\end{cases}, \quad \forall t > 0,
%\end{align*}
we get that $C(t) = \frac{p_{atm}}{\rho}$.

We also make the reasonable assumption that water is incompressible, i.e., $\rho$ is a constant. Equation \eqref{eq:massConservation} reduces to
$\nabla \cdot \bm{v} = \Delta \phi = 0$, which gives that the flow is incompressible. We now have
the equations
\begin{align}
    \label{eq:IncompNavierStokes}
    \begin{cases}
        \Delta \phi = 0,                                                 &    (x,y) \in \Omega(t), \, t > 0 \\
        \phi_t + \frac{1}{2}|\nabla \phi|^2 + \frac{p-p_{atm}}{\rho} - g z = 0,
                                                                         &    (x,y) \in \Omega(t), \, t > 0 \\
        \nabla \phi \cdot \bm{n} = 0,                                    &    \bm{x} \in \Gamma_N \\
        \phi = f,                                                        &    \bm{x} \in \Gamma_D
    \end{cases},
\end{align}
where $\Gamma_N$ are the Neumann boundaries and $\Gamma_D$ the Dirichlet. Naturally, no particles can move through the sea bed. Moreover, we assume no water is leaving our domain through any sides of the boundary.  This gives us the homogeneous Neumann boundaries. Finally we assume that $\phi$ is known at the water surface, giving the Dirichlet boundary. 

We now consider the 2D case, and compute the special cases of equations \eqref{eq:IncompNavierStokes} on the boundaries. 
The boundary condition for the bottom boundary is $\bm{v} \cdot \bm{n} = 0 $ on $z + h(x,y) = 0$. Using the definition of $\phi$ this yields
\begin{equation}
    \label{eq:phit}%\label{2.9a}
    \nabla \phi  \cdot \bm{n} = 0 \text{ for } z + h(x,y) = 0. 
\end{equation}
On the top of the wave, $z = \eta(x,y)$, which implies that $\frac{d\eta}{dt} = v_z = \frac{dz}{dt}$. Taking the derivative of $\eta$ with respect to $t$, this gives for $z = \eta(x,y)$
\begin{equation*}
    \eta_t + \eta_x\frac{dx}{dt} + \eta_y\frac{dy}{dt} = \frac{dz}{dt}. 
\end{equation*}
By using that $\nabla\phi = \bm{v} = [\frac{dx}{dt}, \frac{dy}{dt}, \frac{dz}{dt}]$, we end up with
\begin{align}
    \label{eq:etaEq}%\label{2.9b}
    \eta_t + \nabla\phi\cdot \big(\eta_x, \eta_y, - 1\big) = 0.
\end{align}
Furthermore inserting $z = \eta(x,y)$ into \eqref{eq:IncompNavierStokes} and using that $p = p_{atm}$ in $z = \eta(x,y)$, the second equation yields
\begin{equation}
    \label{eq:phiEq}
    \frac{\partial \phi }{\partial t} + \frac{1}{2}|\nabla \phi |^2 + g\eta = 0.
\end{equation}
%
% Generation of the wave
%
\section{Generation of the wave}
The wave is generated by a rock falling into the water. We want to find the total energy of the rock with mass $m$ and velocity $v_R$ at height $z$. It is assumed that the rock is
perfectly shaped as a rectangular box or cuboid that has height = 750 m and  width = breadth $\approx 268$ m for every cross-section of the rock in the $xy$-plane. This cross-section area
is estimated to be  $A=72000$ m$^2$ and the volume is $V= 54 \cdot 10^6$ m$^3$. We assume the density of the rock to be the same as the density of concrete, that is $\rho_{R}=2400$ kg$/$m$^3$.
We also assume the rock falls 150 m before landing in water.

\subsection{Kinetic energy of rock at impact}
First, we assume that the rock falls down a steep slope of the mountain with loose pebbels underneath working as a conveyor belt. This makes an assumption of zero friction and free fall reasonable.
Then the speed of the rock as it reaches the water will be $v_{R,1}=55$ m/s from $mgh=\frac{1}{2}mv_R^2$. However it is more likely that the rock slides down an inclined plane with friction.
We assume that this plane is angled by $\theta=45^{\circ}$  and the friction coefficient is $\gamma=0.6$, which makes the acceleration $a=g(\sin(\theta) - \gamma cos(\theta)) \approx 2.83$ m/s$^2$.
The length of the inclined plane is $s=212$ m. Thus the velocity of the rock as it reaches water will now be estimated to be $v_{R,2}=\sqrt{2as} \approx 34.6$ m/s.
Using $E =\frac{1}{2}m v_R^2$ for $v_{R,1}$ and $v_{R,2}$, we get a lower and upper estimate for the kinetic energy of the rock as it reaches water, $E_{upper} \approx 2 \cdot 10^{14}$ J and $E_{lower} \approx 7.8 \cdot 10^{13}$ J.

\subsection{Terminal velocity of the rock}
The rock may reach a constant velocity $v_{\infty}$. The rock achieves constant velocity when the sum of the forces acting upon it equals zero. Solving Newton's second law in equilibrium state for
$v_{\infty}$ in presence of buoyancy force, gravity force and drag we get the equation
\begin{align}
    \label{terminal-lign}
    F_g=F_b + D,
\end{align}
where $F_g=V\rho_{R}g$, $F_b= V \rho_{w}g$ and $D=\frac{1}{2}C_d\rho_{w}Av^2$. $C_d$ is the drag coefficient of the rock which for our assumed shape of the rock is $C_d=1.00$ for a smooth rock.
Solving \eqref{terminal-lign} for $v$, we find that $v_{\infty} = 145$ m/s. 

\subsection{Kinetic energy transferred to the wave}
Lastly, we try to estimate the kinetic energy that can be transferred to water. An upper estimate for energy transferred to water is all the energy. This may not be realistic as the rock does not
come to a complete stop when it hits the water. Some energy goes into noise, heat and motion that do not directly add up to the wave. Therefore, we make an upper estimate of 50\% of the energy going into the wave.
It is difficult to give a lower bound, however we assume a lower bound to be 1\%. Using this information the upper and lower estimated energy going to the wave is $E_{lower}=7.8 \cdot 10^{11}$ J and $E_{upper}=10^{14}$ J.

If the speed of the rock as it reaches water is close to the terminal speed, and the height of the mountain is sufficiently small it would be reasonable to additionally try to model the wave from the flux of water
that the rock pushes out underneath itself instead of only the energy from the crash. Also the potential energy of the rock from the surface of the fjord to the bottom is assumed to not add up to the energy of the wave
as water has high heat capacity and the speed of the rock is increasing. 


%
%    Reduction of a linear model
%
\section{Reduction of a linear model} 
\subsection{Undimensionalization}
We wish to undimensionalize the equations \eqref{eq:IncompNavierStokes}, \eqref{eq:phit}, \eqref{eq:etaEq} and \eqref{eq:phiEq}. Let $u^*$ be an unscaled physical quantity and scale such that $u^* = \mathcal{U}u$, where $u$ is the dimensionless scaled variable, and $u^*$ and $\mathcal{U}$ have the same units. We start by scaling the variables, and make some assumptions.

Assume that the waves are small compared to the depth of the fjord so that $\eta^* =\max(\eta)\eta = N \eta$ with $N << 1$. Assume also that the bottom of the fjord is almost constant, without big peaks, so that we can use a constant $h = \max(h^*)$. The following scales are chosen for the rest of the variables

\begin{table}[H]
\begin{center}
\begin{tabular}{l l}
	$x^* = Lx$, where $L$ is the wavelength      & $t^* = Tt =  \sqrt{\frac{L^3}{h^2g}}  t$ \\
	$y^* = By$, where $B$ is also the wavelength & $\phi^*=\Phi \phi$\\
	$z^* = \max(h^*)z = hz$ 					 & $\bm{v}^*=\nabla \phi^*=[\frac{\Phi}{L}\phi_x,\frac{\Phi}{B}\phi_y, \frac{\Phi}{h} \phi_z]=  [\sqrt{gh} \phi_x, \sqrt{gh} \phi_y, \frac{\Phi}{h} \phi_z]$ \\
	$p^*-p_{atm} = -\rho g z^* p =- \rho g z h p$& $\eta^*=N \eta$\\\\
\end{tabular}
\end{center}
\end{table}

The scale for $\bm{v}$ is choosen from the fact that shallow water waves travel with the phase velocity $\sqrt{gh}$. Thus we have the relation $\Phi= L \sqrt{gh}$. The scaling of $T$ will be explained later. 


We start with the first equation in \eqref{eq:IncompNavierStokes} which is $\Delta \phi^* = 0 \text{ for } z^* \in(-h^*, \eta^*)$ and insert the scaled quantities
\begin{align*}
%\frac{\partial^2\phi^*}{\partial x^{*^2}} +  \frac{\partial^2\phi^*}{\partial y^{*^2}} + \frac{\partial^2\phi^*}{\partial z^{*^2}} = 0 \text{ for } z^* \in(-h^*, \eta^*) \\
\frac{\Phi\partial^2\phi}{L^2\partial x^{2}} +  \frac{\Phi\partial^2\phi}{B^2\partial y^{2}} + \frac{\Phi\partial^2\phi}{h^2\partial z^{2}} = 0 \text{ for } zh \in(-h, N\eta).
\end{align*}
We have that $L \sim B \gtrsim h$ and $N\eta \approx 0$. Hence,
\begin{equation}\label{2.8a scaled}
\begin{aligned}
\frac{h^2\phi_{xx}}{L^2} + \frac{h^2\phi_{yy}}{B^2} + \phi_{zz} = 0 \text{ for } z \in(-1,0) \\
\end{aligned}
\end{equation}


Next consider \eqref{eq:phit} which is $
\Phi \nabla \phi \cdot \bm{n} = 0 $ for $ zh + h = 0$. This becomes
\begin{align}\label{2.9a scaled}
%\frac{\Phi \partial \phi}{h \partial z} = 0 \\
\nabla \phi \cdot \bm{n} = 0
\end{align}
for $z +1 = 0$.


Now, we move on to the second equation in \eqref{eq:IncompNavierStokes} and insert the scaled variables and divide by $\frac{\Phi}{T}$ to make the equation non-dimensional. We get
\begin{equation*}
\frac{\partial \phi}{\partial t} + \frac{\Phi T}{2}\Big((\frac{\partial \phi}{L\partial x})^2 + (\frac{\partial\phi}{B\partial y})^2 + (\frac{\partial \phi}{h\partial z})^2\Big) - \frac{\rho g z h p T }{\Phi} + \frac{Tghz}{\Phi} = 0.
\end{equation*}
Now only \eqref{eq:etaEq} and \eqref{eq:phiEq} remains to be scaled. %We scale them, keeping in mind that we want to end up with equations (3.1) in the project description.
To scale equation \eqref{eq:etaEq}, we insert the scaled variables, multiply by $T/N$ and move $\eta_t$ to the left-hand-side. We recieve
\begin{equation} \label{etat}
 \eta_t = T (-\frac{\Phi}{L^2} \phi_x  \eta_x - \frac{\Phi}{B^2} \phi_y \eta_y + \frac{\Phi}{hN} \phi_z ),
\end{equation}
for $z = N\eta \approx 0$.
Inserting the scaled variables in equation \eqref{eq:phiEq} we get
\begin{equation} \label{eq13}
\frac{ h}{g T^2 } \frac{\partial \phi }{\partial t} + \frac{h}{g\Phi T}\Big((\frac{\Phi \phi_x}{L})^2+(\frac{\Phi \phi_y}{B})^2+(\frac{\Phi \phi_z}{h})^2\Big) + \frac{Nh}{\Phi T}\eta = 0,
\end{equation}
for $z = N\eta \approx 0$.


Now we do a partial derivation of \eqref{eq13} with respect to $t$ and substitute $\eta_t$ from equation \eqref{etat} into the new equation. We recieve the following
\begin{equation} \label{ti}
\frac{ h}{ g T^2 } \phi_{tt}+ \frac{h}{g \Phi T} \phi_t\Big((\frac{\Phi \phi_x}{L})^2+(\frac{\Phi \phi_y}{B})^2+(\frac{\Phi \phi_z}{h})^2\Big) + Nh(-\frac{1}{L^2} \phi_x  \eta_x - \frac{1}{B^2} \phi_y \eta_y + \frac{1}{hN} \phi_z ) = 0.
\end{equation}
Thus, we have roughly completed the scaling and udimensionalization of the equations. We should however try to specify the scalars where possible, but we will wait with this until after we have explained what happens when we neglect non-linear terms in the two-dimensional case.


We want to make a linearized model for the wave in the domain and in the boundaries. We first assume that the solution is invariant in the $y$-direction such that we can neglect these terms from our equations. Equation \eqref{2.8a scaled} reduces to 
\begin{equation} \label{3.1a}
\mu \phi_{xx} + \phi_{zz} = 0\text{, where }\mu = h^2/L^2,
\end{equation}
for $x \in \mathbb{R}$ and $z \in (-1,0)$. $\mu$ represents the ratio of the depth of the fjord and the typical wavelenght.  We have assumed that the bottom surface is smooth such that  $\mathbf{n} \approx [0,0,1]$. Then \eqref{2.9a scaled} becomes
\begin{align*}
%\frac{\Phi \partial \phi}{h \partial z} = 0 \\
\phi_z = 0,
\end{align*}
for $z +1 = 0$.
Additionally, we can simplyfy equation \eqref{ti} such that it becomes
\begin{equation} \label{almost3.1c}
\mu_2 \phi_{tt} + \epsilon_1 \frac{\partial }{\partial t} \phi_x^2+ \epsilon_2 \frac{\partial }{\partial t}\phi_z^2 - \epsilon_3 \phi_x  \eta_x +  \phi_z  = 0,
\end{equation}
where $\mu_2=\frac{h}{gT^2}$, $\epsilon_1= \frac{h^2}{L^2} \sqrt{\frac{h}{L}}$, $\epsilon_2=\sqrt{\frac{h}{L}}$ and $\epsilon_3=\frac{Nh}{L^2}$.
From the project description, we know that $\mu$ in \eqref{3.1a} and $\mu_2$ in \eqref{almost3.1c} must be equal. Since $\mu=\frac{h^2}{L^2}$ and $\mu_2=\frac{L}{T^2g}$ this gives $\frac{h^2}{L^2}=\frac{L}{T^2g} \Rightarrow T=\sqrt{(\frac{L^3}{h^2g})}$.
When we neglect the non-linear terms of \eqref{almost3.1c}, that is remove the terms coupled with $\epsilon_1, \epsilon_2, \epsilon_3$, equation \eqref{almost3.1c} becomes
\begin{equation*}
\mu \phi_{tt}+ \phi_z=0.
\end{equation*}
Furthermore it would be reasonable to assume that $0<\epsilon_n<<1$ for $n=1,2,3$, such that neglection of the non-linear terms is reasonable. By linearinzing our scaled equations we have found the scale $T$, and made assumptions for the $\epsilon$'s. We note that we do not have $\eta$ in any of the linearized equations. This is because $\eta$ is expressed through $\phi$ when we combine equations \eqref{eq:etaEq} and \eqref{eq:phiEq}.

\subsection{Solving the linear model}
We now want to solve the linear model, which we summarize as
\begin{equation}
\begin{cases}
\label{eq-ana:sjefsproblemet}
\mu \phi _{xx} + \phi_{zz} = 0, x \in \mathbb{R}, 0 \leq z \leq 1 \\
\phi _z = 0,    & z = -1 \\
\mu \phi_{tt} + \phi_z = 0, & z = 0
\end{cases}
\end{equation}
Using separation of variables, we write $\phi(x,z,t) = X(x)Z(z)T(t)$.

After imposing harmonic boundaries i.e. $\phi(x) = \phi(x+L)$ ($L$ is the distance in direction $x$ in our domain of consideration) we know that $X(x)$ must be harmonic. The same argument goes for $T(t)$. This leads us to the equation
\begin{equation*}
\phi_k (x,z,t) = Z_k(z) \exp \Big( i \big( kx - \omega(k)t \big) \Big)
\end{equation*}
where $k=\frac{2 \pi n}{L}$, a necessity to make $X(x) = X(x+L) = 0$. $k$ can take multiple values as a function of $n$, all of which lead to different solution to our problem \eqref{eq-ana:sjefsproblemet}. Let us start by putting the separated $\phi$ into the primary equation.

\begin{equation}
\mu \phi_{xx} + \phi_{zz} = 0 \Longrightarrow \mu \frac{X '' (x)}{X(x)} = -\frac{Z '' (z)}{Z(z)}
\label{eq-ana:separasjon}
\end{equation}

Which must be constant as both sides of the equation are functions of different variables. Equation \eqref{eq-ana:separasjon}
leads to a second order differential equation that can be solved using characteristic equations
\begin{equation*}
Z''(z) - \mu k^2 Z(z) \Longleftrightarrow r^2 - \mu k^2 = 0.
\end{equation*}
Solving the characteristic equation gives us $r = \pm \sqrt{\mu}k$ meaning
\begin{equation*}
Z(z) = A e^{\sqrt{\mu}k z} + Be^{-\sqrt{\mu}k z}
\end{equation*}

Using the initial condition for $z = -1$ we find an expression for the constant $B$, which inserted into the expression for $Z(z)$ leads to
\begin{equation*}
Z_k(z) = A_k \left( e^{\sqrt{\mu}kz} + e^{ \sqrt{\mu}k(z+2) } \right).
\end{equation*}

The other initial condition (for $z=0$) makes us able to find an expression for the dispersion constant, $\omega(k)$,
\begin{equation}
\mu \phi_{tt}(x,0,t) + \phi_{z}(x,0,t) = 0 \Longleftarrow \mu \frac{T''(t)}{T(t)} = - \frac{Z'(0)}{Z(0)}.
\label{eq-ana:ini2}
\end{equation}
Solving \eqref{eq-ana:ini2} with respect to $\omega(k)$, we find
\begin{equation*}
\omega(k) = \left( \frac{k}{\sqrt{\mu}} \right)^{\frac{1}{2}}
\end{equation*}

We now have an expression for each solution $\phi_k(x,z,t)$,
\begin{equation*}
\phi_k(x,z,t) = A_k \left( e^{\sqrt{\mu}kz} + e^{-\sqrt{\mu}k(z+2)} \right) e^{i\big( kx - \omega(k)t \big)}
\end{equation*}

We may use the superposition principle to assess the solution $\phi(x,z,t)$:
\begin{equation}
\label{eqana:superposition}
\phi(x,z,t) = \sum_{k=0}^{\infty} Z_k e^{i\big(kx - \omega(k) t \big)}
\end{equation}

We need to impose another initial condition, $\phi(x,0,0) = K(x)$ which, when inserted into \eqref{eqana:superposition} becomes

\begin{equation}
\phi_n(x,0,0) = \sum_{n=0}^{\infty} A_n \left( e^{\sqrt{\mu}\frac{2 \pi n}{L}} + 1 \right) e^{ i \big( \frac{\pi n}{L}x \big) } = G(x)
\label{eqana:superduperposition}
\end{equation}

By adjusting the constants $A_n$ we can reduce $X(x)$ from a complex exponential to a sine function when the boundary conditions are imposed. The initial condition is chosen to be a harmonic function, $G(x) = \sin \left(\frac{\pi n}{L}x \right)$. We can now solve \eqref{eqana:superduperposition} wrt $A_n$:

\begin{equation*}
\sum_{n=0}^{\infty} A_n \left( e^{\sqrt{\mu}\frac{2 \pi n}{L}} + 1 \right) \sin \left(\frac{\pi n}{L}x \right) = \sin \left( \frac{\pi }{L}x \right)
\end{equation*}

giving a non-zero result only for $n=1$
\begin{equation*}
A_1 = \frac{e^{- \sqrt{\mu}2\pi  /L } }{1 + e^{- \sqrt{\mu} 2\pi /L}}.
\end{equation*}


Now we can insert this into equation \eqref{eqana:superposition} for the final result:
\begin{equation*}
\phi(x,z,t) = \frac{ e^{ \sqrt{\mu} \frac{\pi }{L}z}  + e^{ -\sqrt{\mu}\frac{\pi }{L}(z+2)    }}{1 + e^{-2\sqrt{\mu}\frac{\pi }{L} }} e^{i \left( \frac{\pi }{L}x - \frac{\pi }{L\sqrt{\mu}} \tanh (\sqrt{\mu}\frac{\pi }{L})t \right)}
\end{equation*}


%\begin{figure}
%    \centering
%    \begin{subfigure}[b]{0.48\textwidth}
%        \includegraphics[width=\textwidth]{matlabcropped.pdf}
%        \vspace{-20pt}
%        \caption{Wave plot}
%        \label{figana:wave}
%    \end{subfigure}
%    ~ %add desired spacing between images, e. g. ~, \quad, \qquad, \hfill etc. 
%      %(or a blank line to force the subfigure onto a new line)
%    \begin{subfigure}[b]{0.48\textwidth}
%        \includegraphics[width=\textwidth]{gradientcropped.pdf}
%        \vspace{-20pt}
%        \caption{Gradient plot}
%        \label{figana:gradient}
%    \end{subfigure}
%    \caption{Resulting plots at $t=0$.}\label{figana:plots}
%\end{figure}


If we instead of an imaginary exponential use the equivalent sine representation (which we get by only considering the imaginary part of $\phi(x,z,t)$), the equation \eqref{eqana:superposition} is expressed as:
\begin{equation*}
\sum_{n=0}^\infty Z_n \sin \left(  kx-\omega(k)t  \right).
\end{equation*}

We need initial condition $\phi(x,0,0) = G(x)$, which will be expressed as

\begin{equation}
\label{eqana:fouriersine}
\phi(x,0,0) = \sum_{n=1}^\infty A_n \left( 1 + e^{-2\sqrt{\mu}\pi/L} \right) \sin \left( \frac{\pi n}{L}x \right)
\end{equation}

We must choose the coefficients, $A_n$ so that \eqref{eqana:fouriersine} reduces to the fourier sine series of $G(x)$,

\begin{equation*}
A_n = \frac{2}{L} \int_0^L G(x) \sin \frac{n \pi x}{L} d x
\end{equation*}
%
%    Inundation
%
\section{Inundation}
%
%    Derivation of the shallow water equations
%
\subsection{Derivation of the shallow water equations}
When the wave approaches the shore, the water depth becomes small. We now want to show that when the water depth is much smaller than the typical wavelength, we can estimate the situation by the following equations 
\begin{align}
    \label{eq:shallow}
    \begin{cases}
	    u_t + \left(\frac{1}{2}\varepsilon u^2 + \eta\right)_x & = 0 \\
	    \eta_t + (u(\varepsilon\eta + h))_x & = 0
	\end{cases},
\end{align}
where $u = \phi_x$. 

In this section we assume the reader is familiar with scaling of variables and will not explain how this is done. To find the first equation in \eqref{eq:shallow} we consider \eqref{eq:phiEq} and take the partial derivative
with respect to $x$. We use scales $T$ for time, $\Phi$ for $\phi$, $X$ for the $x$-direction and $Z$ for the $z$-direction. For $\eta$ we scale with $E$ and for $h$ we scale with $H$. We assume the solution is
invariant in the $y$-direction and use that $\phi_x=u$. Then we get 
\begin{equation*}
\frac{\Phi}{T}\frac{\partial u}{\partial t} + \frac{\Phi^2}{2X^2}\frac{\partial (u^2)}{\partial x}+ \frac{\Phi^2}{2Z^2}\frac{\partial(\phi_z^2)}{\partial x} + g\frac{E \partial \eta}{\partial x} = 0,
\end{equation*}
where $\partial\phi_z^2/\partial x=0$. We multiply by $T$ and divide by $\Phi$ and collect the derivatives with respect to $x$,
\begin{equation}
    \label{shallow2-ferdig}
    u_t + \left(\frac{T\Phi}{2X^2}u^2 + \frac{gTE}{\Phi}\eta\right)_x = 0.
\end{equation}
From \eqref{shallow2-ferdig} we see that $E = \frac{\Phi}{T g}$ and $\varepsilon =\frac{\Phi T}{X^2}$, and we have derived the first equation in \eqref{eq:shallow} for scaled variables.

Next we want to derive the second equation of \eqref{eq:shallow}. To do this, we start with \eqref{eq:etaEq} in $z = \eta$. When scaled this is 
\begin{equation}
    \label{eta_t..}
    \eta_t + \frac{\Phi T}{X^2}\phi_x\eta_x - \frac{\Phi T}{ZE}\phi_z = 0,
\end{equation}
and see that we need an expression for $\phi_z$ in in $z = \eta$. We integrate the first equation of \eqref{eq:IncompNavierStokes} from $-h$ to $\eta$ and find 
\begin{equation*}    
    \frac{1}{X^2}\phi_{xx}(E\eta + Hh) + \frac{1}{Z}\phi_z|_{z=\eta} - \frac{1}{Z}\phi_z|_{-h} = 0.
\end{equation*}
We also use from \eqref{eq:phit} that in $z = -h$ we can use $\phi_z = -\phi_x\frac{n_x}{n_z} \approx -\phi_x h_x$. We now have an expression for $\phi_z$ in $z = \eta$ which is when scaled
\begin{equation*}
    \frac{1}{Z}\phi_z = - \frac{H}{X^2}\phi_x h_x - \frac{1}{X^2}\phi_{xx}(E\eta + Hh).
\end{equation*}
We insert this into \eqref{eta_t..} and get
\begin{equation*} 
    \eta_t + \frac{\Phi T}{X^2}\phi_x\eta_x + \frac{T\Phi H}{EX^2}\phi_x h_x + \frac{T\Phi}{X^2}\phi_{xx}(\eta + h\frac{H}{E}) = 0.
\end{equation*}
If we let $\varepsilon = \Phi T/X^2 $ as earlier and also let $\varepsilon = E/H$, then we get
\begin{equation*}
\eta_t + (u(\varepsilon\eta + h))_x = 0,
\end{equation*}\\

for the scaled variables, which is what we wanted to show. $\varepsilon$ turns out to be the scale of $\eta$ over the scale of $h$.
%
%    Transforming the Shallow Water Equations
%
\subsection{Transforming the Shallow Water Equations}
We now want to decouple the \emph{variables} of the shallow water equations \eqref{eq:shallow}. To this end, 
assume that the bottom has a constant slope, $h_x = a$ (note: we allow $a$ = 0). Further set $\varepsilon =1$,
and let $\eta$ be the total wave height, that is $\eta = \varepsilon\eta + h$ from Equation \eqref{eq:shallow}.
Then the shallow water equations can then be written in matrix form as

\begin{equation*}
\begin{bmatrix}
	\eta \\
	u
\end{bmatrix}_t
+
\begin{bmatrix}
u	&& 	\eta \\
1 	&&	u
\end{bmatrix}
\begin{bmatrix}
\eta\\
u
\end{bmatrix}_x
= 
\begin{bmatrix}
0\\
a
\end{bmatrix}
\end{equation*}
The matrix has eigenvalues $\lambda_1 = u + \sqrt{\eta}$ and $\lambda_2 = u - \sqrt{\eta}$, with eigenvectors $v_1 = (1,1/\sqrt{\eta})^T$ and $v_2 = (1, {-}1\sqrt{\eta})^T$ respectively. This gives the two Riemann invariants
\begin{align*}
	\begin{aligned}
		V = u + \int \frac{1}{\sqrt{\eta}} = u + 2\sqrt{\eta},\quad
		W = u - \int \frac{1}{\sqrt{\eta}} = u - 2\sqrt{\eta}.
	\end{aligned}
\end{align*}
This gives us
\begin{align}
\begin{aligned}
	\label{eq:shallowWaterRiemannV}
	V_t + \lambda_1V_x &= \left(u_t + \frac{\eta_t}{\sqrt{\eta}}\right) + (u + \sqrt{\eta})\left(u_x + \frac{\eta_x}{\sqrt \eta}\right)\\
	& = u_t + \left(\frac{1}{2}u^2 + \eta\right)_x + \frac{1}{\sqrt \eta}(\eta_t + (u\eta)_x) = a,
\end{aligned}
\end{align}
and
\begin{align}
	\label{eq:shallowWaterRiemannW}
	\begin{aligned}
		W_t + \lambda_2W_x &= \left(u_t - \frac{\eta_t}{\sqrt{\eta}}\right) + (u - \sqrt{\eta})\left(u_x - \frac{\eta_x}{\sqrt \eta}\right)\\
		& = u_t + \left(\frac{1}{2}u^2 + \eta\right)_x - \frac{1}{\sqrt \eta}(\eta_t + (u\eta)_x) = a.
	\end{aligned}
\end{align}
\subsection{Analytical solution}
We assume $W_x(0,x) = 0$. If we differentiate $\omega = W_x$ with respect to $x$, then

\begin{equation}
	\frac{d \omega}{d x} = \omega_t + \left( \lambda_2 \omega_x \right) = 0
\end{equation}
The means that $\omega$ is constant, and using the initial condition $W_x(0,x) = 0$ we get $\omega(t,x) = W_x(t,x) = 0$.


Since $W_x = 0$, equation \eqref{eq:shallowWaterRiemannW} gives us $a = W_t$. We can then set up  the following identities
\begin{align}
\left(\frac{V-W}{4}\right)_x = \frac{V_x}{4} = \left( \sqrt{\eta}\right)_x,& \quad
\left(\frac{V+W}{2}\right)_x = \frac{V_x}{2} = u_x \label{eqana:id1}\\ 
\left(\frac{V-W}{4}\right)_t = \frac{V_t - a}{4} = \left( \sqrt{\eta}\right)_t,  &\quad \left(\frac{V+W}{2}\right)_t = \frac{V_t + a}{2} = u_t.
\label{eqana:id3}
\end{align}
From Equation \eqref{eqana:id1} we get $u_x = 2\left( \sqrt{\eta}\right)_x$, and from Equation \eqref{eqana:id3} we get $2u_t  =\left( \sqrt{\eta}\right)_t + a$. Solving this system of equation we obtain $u = 2\sqrt{\eta}+at+ C$, where $C$ is a constant. Setting $t=0$, we see that $C = u(0,x)-2\sqrt{\eta}$. This is valid for any $x$, which impose a restriction on the initial conditions. Inserting $u$ into the expression for $\lambda_1$, gives us $\lambda_1 = 3 \sqrt{\eta} + at+C$. Rearranging the terms in \eqref{eq:shallowWaterRiemannV}, and using the identities \eqref{eqana:id1} and \eqref{eqana:id3} yields
\begin{equation}
	\left(\sqrt{\eta}\right)_t + \left( 3 \sqrt{\eta} + at + C \right) \left(\sqrt{\eta}\right)_x = 0.
\end{equation}



Using the method of characteristics we can set up the system of equations 
\begin{equation*}
\label{blablabla}
\begin{cases}
\frac{d \sqrt{\eta(t,x(t))}}{dt} = 0\\
\frac{d x(t)}{d t} = 3 \sqrt{\eta} + at + u(0,x(0)) - 2\sqrt{\eta (0,x(0))} .
\end{cases}
\end{equation*}
Solving this system yields that $\eta$ is constant along the characteristic lines
\begin{equation}
	x(t) = 3 \left(\sqrt{\eta\left(x(0),0\right)}+ u(0,x(0)) - 2\sqrt{\eta (0,x(0))} \right)t + \frac{a}{2}t^2 + x(0).
\end{equation}

The method of characteristics tells us that $\eta$ remains constant as the wave reaches the shore and the seafloor rises. This would indicate that any wave would eventually gain an amplitude equal to the depth of the sea as it reaches the shore. Needless to say, this does not correspond well with our numerical results. As it depends on unreasonable assumptions it serves merely as a proof of concept rather than an accurate solution to our problem.


%
%    Numerics
%


\section{Numerics}

Based preceding, we now present three different numerical schemes for solving the Tsunami problem.
%
%    Mimectic finite differences/forward difference shceme
%
\subsection{Mimectic finite differences/forward difference scheme}

The idea is as follows: We will use Mimetic Finite Differences (MFD) \cite{raynaud15} to solve the first equation of \eqref{eq:IncompNavierStokes}, assuming we initially know
$\phi$ at the top, along with $\eta$. Next, we use this solution in a forward-difference scheme to obtain the next $\eta$, and $\phi$. This will give a Mimetic finite
difference/forward difference scheme, abbreviated M(FD)$^2$S.

In order to obtain a discrete formulation of \eqref{eq:IncompNavierStokes}, we introduce the following notation:
We assume the whole domain $\Omega$ is discretized into a union $\mathcal{T}_h$ of $N^x \times N^y$ adjecent, non-overlapping
polygons with four edges. Further, we consider the top faces, at $z = \eta$, numbered from $1$ (left) to $N^x$ (right), and denote
the centroid of face $i$ by $x_i$, so that $x_1 < x_2 < \dots < x_{N^x}$. We denote the distance between each centroid by
$h_i := x_{i+1}-x_i$. Morover, we disctretize time interval $[0, T]$ as $0 = t_0 < t_1 < \cdots < t_{N^t}$, where $t_{n+1}-t_n := k$ for all $n$.
For a function $g(x,t)$, we can now denote $g(x_i, t_n) = g_i^n$, and $\bm{g}^n = \Big(g_1^n, \dots, g_{N^x}^n\Big)^\top$, and its partial derivative with respect
to a variable $\xi$ as $g_\xi(x_i, t_n) = g_{\xi}|_i^n$, and $\bm{g}_\xi^n = \Big(g_{\xi}|_1^n, \dots, g_{\xi}|_{N^x}^n\Big)^\top$.

To approximate the derivative of a function $g$ with respect to $t$, we use Taylors formula to obtain forward differences:
\begin{align}
    \label{eq:ForwardDiff}
    \begin{aligned}
	    g(x_i, t_{n+1})       & = g(x_i, t_n) +  (t_{n+1}-t_n)g_t(x_i, t_n) + \mathcal{O}\big((t_{n+1}-t_n)^2\big) \\
	    \Rightarrow g_t|_i^n & = \frac{g_i^{n+1}-g_i^n}{k} + \mathcal{O}(k).
	\end{aligned}
\end{align}

To approximate its derivative with respect to $x$, we use central differences:
\begin{align*}
    g(x_{i+1}, t_n)       & = g(x_i, t_n) +  (x_{i+1}-x_i)g_x(x_i, t_n) + \mathcal{O}\big((x_{i+1}-x_i)^2\big) \\
    g(x_{i-1}, t_n)       & = g(x_i, t_n) +  (x_{i-1}-x_i)g_x(x_i, t_n) + \mathcal{O}\big((x_{i-1}-x_i)^2\big) \\    
    \Rightarrow g_x|_i^n & = \frac{g_{i+1}^n-g_{i-1}^n}{h_i + h_{i-1}} + \mathcal{O}(h_i) + \mathcal{O}(h_{i-1}).
\end{align*}
At the end points, we need to approximate the derivatives by forward differences. In particular, following \eqref{eq:ForwardDiff}, 
we obtain
\begin{align*}
    g_x|_1^n = \frac{g_2^n -g_1^n}{h_1} + \mathcal{O}(h_1), \quad g_x|_{N^x}^n = \frac{g_{N^x}^n -g_{N^x-1}^n}{h_{N^x-1}} + \mathcal{O}(h_{N^x-1}).
\end{align*}

We now make the following sin: We assume that $\nabla \phi$ changes sufficiently slow with each time step, such that we can use $\nabla \phi^n$
in place of of $\nabla \phi^{n+1}$. Hence, we can formulate \eqref{eq:etaEq} as
\begin{align*}
    \frac{\eta_i^{n+1} - \eta_i^n}{k} + \phi_x|_i^{n} \frac{\eta_{i+1}^{n+1}-\eta_{i-1}^{n+1}}{h_i + h_{i-1}} - \phi_z|_i^n
                       = \mathcal{O}(k) + \mathcal{O}(h_i) + \mathcal{O}(h_{i-1}) \xrightarrow{N^t, N^x \rightarrow \infty} 0,
\end{align*}
which yields the matrix system
\begin{align}
    \label{eq:etan+1}
    \bm{A} \bm{\eta}^{n+1} = \frac{1}{k}\bm{\eta}^n + \bm{\phi}_z^n,
\end{align}
where $\bm{A}$ is a tridiagonal matrix with the following diagonals:
\begin{align*}
    \text{Super diagonal: } &\frac{1}{2} \bigg(\frac{\phi_x|_2^{n}}{h_2 + h_1}, \frac{\phi_x|_3^{n}}{h_3+h_2},
                                        \dots, \frac{\phi_x|_{N^x-1}^n}{h_{N^x-1} + h_{N^x-2}}, 2\frac{\phi_x|_{N^x}^n}{h_{N^x-1}}\bigg) \\
    \text{Main diagonal: }  &\frac{1}{k}  \Big(1, \dots, 1\Big)                                        \\
    \text{Sub diagonal: }  -&\frac{1}{2} \bigg(2\frac{\phi_x|_1^{n}}{h_1}, \frac{\phi_x|_2^{n}}{h_2+h_1}, \frac{\phi_x|_3^n}{h_3 + h_2},
                                        \dots, \frac{\phi_x|_{N^x-1}^n}{h_{N^x-1} + h_{N^x-2}}\bigg) \\
\end{align*}
Following the same procedure as for \eqref{eq:ForwardDiff}, we obtain the discretized version of \eqref{eq:phiEq} in matrix form:
\begin{align}
    \label{eq:phin+1}
    \bm{\phi}^{n+1} = \bm{\phi}^n - k\bigg(\frac{1}{2}\bm{\xi}^n + g \bm{\eta}^{n+1}\bigg), \quad \bm \xi ^{n} = \left( |\nabla \phi_1^n|^2,\ldots , |\nabla_{N^x}^n|^2 \right)^\top.
\end{align}
 

We note that we will need $\nabla \phi$ on each top face. To this end, we note that MFD gives us the fluxes on each half-face, $v_{f} \approx \int_{f} \nabla \phi \cdot n_{f} \, \d s$.
Moreover, we know that any vector $\bm{v} \in \mathbb{R}^2$ can be written as $(\bm{v}\cdot\bm{n}_1) \bm{n}_1 + (\bm{v}\cdot\bm{n}_2) \bm{n}_2$,
where $\bm{n}_1$ and $\bm{n}_2$ are two orthogonal vectors of unit length.
Consider $\nabla \phi$ on a top half-face $f_t$, with corresponding right and left half-faces $f_r$ and $f_l$ sharing the same cell.
If we are not altering the grid in the $x$-direction, we can then approximate $\nabla \phi|_{f_t}$ as
\begin{align*}
    \nabla \phi|_{f_t} \approx |f_t|^{-1} v_{f_t} \bm{n}_{f_t} + \frac{1}{2}(|f_r|^{-1}v_{f_r}-|f_l|^{-1}v_{f_l})\big(\bm{n}_{f_r} - (\bm{n}_{f_r} \cdot \bm{n}_{f_t})\bm{n}_{f_t}\big)
\end{align*}

We then choose $\nabla \phi|_{f_t}$ to be our approximation of $\nabla \phi$ at the centroid of $f_t$.

A natural next step is now to implement the same procedure in three dimensions. To simplify the discretization, we first calculate
the gradient of $\eta$ numerically, and then solve for $\eta^{n+1}$ explicitly. In matrix notation, we have
\begin{align}
    \label{eq:etan+13d}
    \bm{\eta}^{n+1} = \bm{\eta}^n - k \big(\bm{\phi}_x^n, \bm{\phi}_y^n, \bm{\phi}_z^n\big)\cdot(\bm{\eta}_x^n, \bm{\eta}_y^n, -\bm{1}\big),
\end{align}
where the dot product matrices is understood to be the dot product of the rows forming each matrix.

We denote the domain $\Omega^n := \Omega(t_n)$, and define $\mathcal{T}_h^n$ to be the discretization of $\Omega^n$
at a time $t_n$. Moreover, we denote the discrete solution $\phi$ on $\mathcal{T}_h^n$ at time $t_n$ by $\bm{\Phi}^n$.

We can now state an algorithm for solving the system \eqref{eq:IncompNavierStokes}. It is given in Algorithm \eqref{alg:IncomNavierStokes}.
%
%    Algorithm
%
\begin{algorithm}
    \caption{M(FD)$^2$S}
    \begin{algorithmic}[1]
    \State    \textbf{Input:} $\bm{\eta}^0$, $\bm{\phi}^0$, $\mathcal{T}_h^0$.
        \For    {$n \leftarrow 0,N^t$}
	    \State    Change $\mathcal{T}_h^n$ according to $\bm{\eta}^n$.
            \State    Solve the first equation of \eqref{eq:IncompNavierStokes} with $f = \bm{\phi}^n$ to obtain
                      $\bm{\Phi}^{n}$, $\bm{\phi}_x^n$ and $\bm{\phi}_z^n$ using MFD.
			\If {Two dimensions}
                \State    Calculate $\bm{\eta}^{n+1}$ from \eqref{eq:etan+1}
			\ElsIf {Three dimensions}
			    \State    Calculate $\bm{\eta}^{n+1}$ from \eqref{eq:etan+13d}
			\EndIf
        \State    Calculate $\bm{\phi}^{n+1}$ from \eqref{eq:phin+1}
        \EndFor
	\end{algorithmic}
	\label{alg:IncomNavierStokes}
\end{algorithm}
%
%    A method of characteristics    
%
\subsection{A method of characteristics}
Assume a flat bottom, i.e, $a = 0$. We present a scheme for solving the coupled equations \eqref{eq:shallowWaterRiemannV} and \eqref{eq:shallowWaterRiemannW}. Along each characteristics, $x^V(t)$ and $x^W(t)$, $V$ and $W$ are constants. For small time steps we assume $\lambda_1$ and $\lambda_2$ constant, which gives the linear characteristics
\begin{align}
	\label{eq:characteristicEqVandW}
	x^V(t^V_0+\Delta t_V) = x^V_0 + \lambda_1(x_0,t^V_0)\Delta t_V ,\quad x^W(t^V_0+\Delta t_W) = x^W_0 + \lambda_2(x_0,t_0^W)\Delta t_W.
\end{align}
These two lines cross each other when 
\begin{equation}
	\label{eq:characteristicCross}
	\Delta t_v = \frac{x_0^W - x_0^V + \lambda_2(x_0^W,t_0^W)(t_0^V - t_0^W)}{\lambda_1(x_0^V,t_0^V)+\lambda_2(x_0^W,t_0^W)}
\end{equation}

Using this present a natural numerical solution given in  Algorithm \ref{alg:ShallowWaterConstSeaBed}.  
For the characteristics $x^V(t)$ and $x^W(t)$ starting in the same point $(x_i,t_i)$ we know that $\frac{\d x^V}{\d t}\geq \frac{\d x^W}{\d t}$ since $\lambda_1(x_i,t_i)\geq \lambda_2(x_i,t_i)$, so the characteristic of $V_i^j$ will always cross the characteristic of $W_{i+1}^j$ before any other. 
\begin{algorithm}
	\caption{Shallow Water Constant Sea Bed}
	\begin{algorithmic}[1]
		\State    Input:$\{(x^0_i,t^0_i)\}_{i=1,\ldots n}$, $\eta^0_i=\eta(x^0_i,t^0_i)$, $u^0_i = u(x_i^0,t^0_i)$,  $T$.

		\State $j\leftarrow 0$
		\While    {$\max \{t^j\}_{i=1,\ldots,n} < T$}
		\State $V^j_i \leftarrow u_i + 2\sqrt{\eta_i}$
		\State $W^j_i \leftarrow u_i - 2\sqrt{\eta_i}$
		\State From each point $(x^j_i,t^j_i)$, calculate the crossing points using 		
		\eqref{eq:characteristicEqVandW} and \eqref{eq:characteristicCross}
		\State Set $(x^{j+1}_i,t^{j+1}_i)$ equal to the crossing points.
		\State $\eta_i^{j+1} \leftarrow \left(\frac{V_i^j -W_i^j}{4}\right)^2$
		\State $u_i^{j+1} \leftarrow V - 2\sqrt{\eta_i^{j+1}}$
		\State $j\leftarrow j + 1$
		\EndWhile
	\end{algorithmic}
	\label{alg:ShallowWaterConstSeaBed}
\end{algorithm}
%
%    A fast and stable implementation
%
\subsection{A fast and stable implementation}
Finally, we proceed to implement the emph{fast and stable well-balanced scheme}, abbreviated WBS$^2$, described in \cite{audusse04}. The numerical flux $\mathcal{F}$ was approximated by
\begin{align*}
    \mathcal{F}(U, W) = \frac{1}{2}\big(F(U) + F(W)\big) + \frac{1}{2} \frac{\Delta x}{\Delta t} (U - W),
\end{align*}
where $F(U) = \big(hu, hu^2 + \frac{1}{2}gh^2\big)^\top$.
%
%    Numerical experiments
%
\section{Numerical experiments}
\subsection{Arrival time and height of the wave}
We have chosen to simulate two real-life scenarios. They are the following:

\textbf{Hellesylt:} The wave travels $12$ km from Åkerneset to Hellesylt. The fjord depth is assumed to be at 300 m from 0 to 7 km, before it ascends linearly
to 30 m at Hellesylt. The fjord width is assumed to be constant at 2 km. The initial energy of the wave is assumed to be of magnitude $10^{13}$ J. This is incorporated in the simulation by
setting the initial wave profile to a bell curve such that it has a potential energy of $10^{13}$ J, and setting $\phi^0$ to zero.

\textbf{Geiranger:} The wave travels $23$ km from Åkerneset to Geiranger. The fjord depth is assumed to be at 300 m from 0 to 7 km, before it ascends linearly
to 30 m at Geiranger. The fjord width is assumed to be constant at 2 km. The initial energy is set to be $10^{13}$J, which is incorporated as in the latter case.

The results are presented in Table \ref{tab:TimeHeight}. Videos of the simulations can be found in \cite{hellesyltMFDFD}, \cite{hellesyltWBSS}, \cite{geriangerWBSS} and \cite{geirangerMFDFD},
\begin{table}[H]
	\begin{center}
	    \begin{tabular}{l r r r r r r}
	                     &    $\quad$    &    \multicolumn{2}{c}{M(FD)$^2$S}    &    $\quad$    &    \multicolumn{2}{c}{WBS$^2$}      \\
	        Case         &               &    $T$ [s]    &    $\eta$ [m]        &               &    $T$ [s]    &    $\eta$ [m]    \\
            \hline                                                                               
            Hellesylt    &               &    260        &    18                &               &    123        &    123               \\
            Geiranger    &               &    560        &    12                &               &    123        &        123               \\
	    \end{tabular}
	\end{center}
	\caption{Numerical results for the arrival time and wave height at Hellesylt and Geiranger}
	\label{tab:TimeHeight}
\end{table}
\subsection{Run-up}
Finally, we exploit the nature of the WBS$^2$ method to approximate the run-up, that is, the highest the wave reaches as it rinses up the mountain side. To this end, we
approximate the ground in the fjord and up the hillside in Geiranger by the parabola $\frac{50}{1000^2}x^2$, where $x$ is the distance in meters from $1000$ m out in the fjord. The initial condition was found from the
M(FD)$^2$ simulation of the Geiranger case.
The resulting run-up was almost 60 m. A video of the simulation can be found in \cite{geirangerUnUpWBS}.    
\subsection{3D simulation}
The M(FD)$^2$S method was also implemented in three dimensions in the unit cube. A video of the simulation
can be found in \cite{MFDFD3dExample}.
%
%    Final remarks
%
\section{Conclusion}
%Even though we have made several assumptions throughout this study, especially for the energy in the wave (+ andre ting?), we have been able to give good estimates of the creation,
%propagation and run-up of the wave. The estimated time for the wave to reach  the populated area is 3.5$-$4 minutes. We have also estimated that the wave will be around 20 m high when it reaches the populated area. 

The Tsunami wave resulting from the Åkerneset rock slide has been analyzed using analytical tools, mathematical modeling, and numerical simulations. In the modeling part of the problem it has been made assumptions about the size and shape of the rock, and how the landslide will be regarding height of the fall and friction on the mountain slope. This of course brings uncertainties to the results for energy transferred into the wave. 

The analytical results of (REFERER til oppdatert løsning) tells us that the wave height $\eta$, from top of the wave to the bottom of the sea, remains constant for all characteristics. This means that as the seafloor ascends, the height of the wave relative to the sea level will increase by the same amount.

Lower and upper bound for the energy transformed from the rock to the water was found to be $8\times 10^{11}$ J and $1\times 10^{14}$ J, respectively. 

Numerical experiments agrees with the results in \citep{harbitz14}. The biggest difference is in the wave height just before it hits land. Some reasons for this may be that we use different initial conditions and different 
bottom topology.

Further work would be to implement the M(FD)$^2$S method for general grids, and use it for a full scale 3D simulation.


\clearpage
\pagenumbering{gobble}
\bibliographystyle{plain}
\bibliography{refs}  

\end{document}
