% Chapter 1

\chapter{Introduction} % Main chapter title

\label{Chapter1} % For referencing the chapter elsewhere, use \ref{Chapter1} 

%----------------------------------------------------------------------------------------

% Define some commands to keep the formatting separated from the content 
\newcommand{\keyword}[1]{\textbf{#1}}
\newcommand{\tabhead}[1]{\textbf{#1}}
\newcommand{\code}[1]{\texttt{#1}}
\newcommand{\file}[1]{\texttt{\bfseries#1}}
\newcommand{\option}[1]{\texttt{\itshape#1}}

%----------------------------------------------------------------------------------------

In this modelling project, we will address three major phases in the development of a tsunami:
\begin{enumerate}[label = \emph{(\roman*)}]
    \item    The creation of a tsunami wave, when the kinetic energy of a falling rock is transferred to the
             water,
    \item    The propagation of the wave in the parts of the fjord where the bottom as approximately constant
             elevation,
    \item    The run-up of the wave when it reaches the end of the fjord and the height of the wave starts
             increasing at the same time that its speed decreases.
\end{enumerate}

In the case of a tsunami event close to a populated area, two questions are of particular interest:
\begin{enumerate}[label = \emph{(\roman*)}]
    \item    How long does it take for the wave to reach the populated area?
    \item    What is the height of the wave when it reaches the shore?
\end{enumerate}

Modeling tools in order to describe the problem, derive the modeling equa-
tions and finally simplify them by only retaining the processes that matter
for a given application (questions of type mod ).
(2) Analytical tools in order to solve simple equations and, in that way, gain
more insight into the governing equations (questions of type ana).
(3) Numerical tools in order to set up and implement numerical methods which
will enable you to treat more realistic cases (questions of type num).